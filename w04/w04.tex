\documentclass{article}

\usepackage[final]{style}
\usepackage[utf8]{inputenc} % allow utf-8 input
\usepackage[T1]{fontenc}    % use 8-bit T1 fonts
\usepackage{hyperref}       % hyperlinks
\usepackage{url}            % simple URL typesetting
\usepackage{booktabs}       % professional-quality tables
\usepackage{amsfonts}       % blackboard math symbols
\usepackage{nicefrac}       % compact symbols for 1/2, etc.
\usepackage{microtype}      % microtypography
\usepackage{verbatim}
\usepackage{graphicx}       % for figures
\usepackage{amsmath}        % equation environments and so on
\usepackage{framed,color}
\definecolor{shadecolor}{rgb}{0.9,0.9,0.9}

\title{CS 6101 Week \#4 Notes: Actor-Critic Introduction, Value Functions and Q-Learning}

\author{
  Note taking: Alexandre Gravier, Joel Lee\\
  \LaTeX{} transcription: Alexandre Gravier <\texttt{al.gravier@gmail.com}> \\
}

\begin{document}

\maketitle

\tableofcontents

\section{Recap about policy gradients}


We define $J(\theta) \doteq E_{\tau\sim p_\theta(\tau)}\left[\sum_t r\left(\mathbf{s}_t,\mathbf{a}_t\right)\right]$ so that the objective of RL can be defined as an optimization exercise consting in finding an assignment of policy parameters $\theta^\star = \arg\max_\theta E_{\tau\sim p_\theta(\tau)}J(\theta)$.

$J(\theta)$ is not usually optimizable as such due to f.e. dimensionality issues, so we
use a sample-based unbiased estimate: $J(\theta) \approx \frac{1}{N} \sum_i \sum_t r\left(\mathbf{s}_{i,t},\mathbf{a}_{i,t}\right)$. Taking the gradient of $J(\theta)$ along $\theta$ allows maximizing the expected reward as per the policy.

\subsection{Policy differentiation with a ``convenient identity''}

Let $r(\tau) \doteq \sum_{t=1}^T r\left(\mathbf{s}_t,\mathbf{a}_t\right)$ the total reward of a trajectory $\tau$.

\begin{subequations}
  \begin{align}
    \nabla_\theta J(\theta) 
      &= \nabla_\theta E_{\tau\sim p_\theta(\tau)}[r(\tau)] & \\
      &= \nabla_\theta\int\pi_\theta(\tau) r(\tau)d\tau &\text{by definition of expectation}\\ 
      &= \int\nabla_\theta\pi_\theta(\tau) r(\tau)d\tau &\text{by linearity} \label{eq:gradjintegral}
  \end{align}
\end{subequations}

At this point, the expression $\int\nabla_\theta\pi_\theta(\tau) r(\tau)d\tau$ seems rather intractable. This is where the following ``convenient identity'' can be used to derive a tractable expression of $\nabla_\theta J(\theta)$.

\begin{shaded}
  \textbf{A convenient identiy}
  \begin{equation} \label{eq:convenientidentity}
    \pi_\theta (\tau)\nabla_\theta \log \pi_\theta (\tau) = \pi_\theta (\tau) \frac{\nabla_\theta \pi_\theta (\tau)}{\pi_\theta (\tau)} = \nabla_\theta \pi_\theta (\tau)
  \end{equation}
\end{shaded}

Furthermore, we can expand the definition of $\pi_\theta\left(\tau\right)$ and take its logarithm:

\begin{subequations}
  \begin{align}
    \pi_\theta\left(\tau\right)
      &= p\left(\mathbf{s}_1\right)\prod_{t=1}^T\pi_\theta\left(\mathbf{a}_t\mid \mathbf{s}_t\right)p\left(\mathbf{s}_{t+1} \mid \mathbf{s}_t ,\mathbf{a}_t\right)\\
    \Leftrightarrow \log\pi_\theta\left(\tau\right)
      &= \log p\left(\mathbf{s}_1\right) + \sum_{t=1}^T\log\pi_\theta\left(\mathbf{a}_t\mid \mathbf{s}_t\right)+ \log p\left(\mathbf{s}_{t+1} \mid \mathbf{s}_t ,\mathbf{a}_t\right) \label{eq:logpitraj}
  \end{align}
\end{subequations}

Using \eqref{eq:convenientidentity} and \eqref{eq:logpitraj} in \eqref{eq:gradjintegral}, the gradient of the objective becomes:

\begin{subequations}
  \begin{align}
    \nabla_\theta J(\theta) 
      &= \int\pi_\theta(\tau)\nabla_\theta\log\pi_\theta(\tau) r(\tau)d\tau \qquad\text{using \eqref{eq:convenientidentity}}\\
      &= E_{\tau\sim p_\theta(\tau)}\left[\nabla_\theta\log\pi_\theta(\tau) r(\tau)\right]\qquad\text{by definition of expectation}\\
      &= E_{\tau\sim p_\theta(\tau)}\left[\nabla_\theta \left[\log p\left(\mathbf{s}_1\right) + \sum_{t=1}^T\log\pi_\theta\left(\mathbf{a}_t\mid \mathbf{s}_t\right)+ \log p\left(\mathbf{s}_{t+1} \mid \mathbf{s}_t ,\mathbf{a}_t\right)\right] r(\tau)\right] \label{eq:gradjlogsum}
  \end{align}
\end{subequations}

We note that in the expression $\nabla_\theta \left[\log p\left(\mathbf{s}_1\right) + \sum_{t=1}^T\log\pi_\theta\left(\mathbf{a}_t\mid \mathbf{s}_t\right)+ \log p\left(\mathbf{s}_{t+1} \mid \mathbf{s}_t ,\mathbf{a}_t\right)\right]$ of the gradient w.r.t. $\theta$ in \eqref{eq:gradjlogsum}, the terms $\log p\left(\mathbf{s}_1\right)$ and $\log p\left(\mathbf{s}_{t+1} \mid \mathbf{s}_t ,\mathbf{a}_t\right)$ are independent of $\theta$, so we are left with:

\begin{subequations}
  \begin{align}
    \nabla_\theta J(\theta) 
      &= E_{\tau\sim p_\theta(\tau)}\left[\nabla_\theta \left[\sum_{t=1}^T\log\pi_\theta\left(\mathbf{a}_t\mid \mathbf{s}_t\right)\right] r(\tau)\right]\\
      &= E_{\tau\sim p_\theta(\tau)}\left[ \left(\sum_{t=1}^T\nabla_\theta\log\pi_\theta\left(\mathbf{a}_t\mid \mathbf{s}_t\right)\right) r(\tau)\right] &\text{by linearity}\\
      &= E_{\tau\sim p_\theta(\tau)}\left[ \left(\sum_{t=1}^T\nabla_\theta\log\pi_\theta\left(\mathbf{a}_t\mid \mathbf{s}_t\right)\right) \sum_{t=1}^T r\left(\mathbf{s}_t,\mathbf{a}_t\right)\right] &\text{by definition of }r(\tau) \label{eq:gradjlog}
  \end{align}
\end{subequations}

In Equation \eqref{eq:gradjlog}, the gradient of $J$ is now a computable function of $\pi_\theta$ only.

We earlier mentioned the sample estimate of $J(\theta) \approx \frac{1}{N} \sum_i \sum_t r\left(\mathbf{s}_{i,t},\mathbf{a}_{i,t}\right)$; similarly $\nabla_\theta J(\theta)$ is approximated with samples, leading us to the algorithm:

\begin{shaded}
  \begin{equation}
    \textbf{\textsc{reinforce} algorithm:}\left\{\begin{array}{l}
      \text{sample } \left\{\tau^i\right\} \text{ from } \pi_\theta\left(\mathbf{a}_t\mid \mathbf{s}_t\right)\\[2mm]
      \nabla_\theta J(\theta) 
        \approx \frac{1}{N} \sum_{i=1}^N\left(\left(\sum_{t=1}^T\nabla_\theta\log\pi_\theta\left(\mathbf{a}_t\mid \mathbf{s}_t\right)\right) \sum_{t=1}^T r\left(\mathbf{s}_t,\mathbf{a}_t\right)\right) \label{eq:gradjapprox}\\[3mm]
        \theta \leftarrow \theta + \alpha\nabla_\theta J(\theta)
    \end{array}
    \right.
  \end{equation}
\end{shaded}

In \eqref{eq:gradjapprox}, there is no use of the Markov property, so the algorithm can be used as such on POMDPs.

\subsection{The bad news}

We introduce some simplifying notation:

\begin{subequations}
  \begin{align}
    \nabla_\theta J(\theta) 
        &\approx \frac{1}{N} \sum_{i=1}^N\left(\left(\sum_{t=1}^T\nabla_\theta\log\pi_\theta\left(\mathbf{a}_t\mid \mathbf{s}_t\right)\right) \sum_{t=1}^T r\left(\mathbf{s}_t,\mathbf{a}_t\right)\right)\\
        &\approx \frac{1}{N} \sum_{i=1}^N\nabla_\theta\log\pi_\theta\left(\tau\right)r\left(\tau\right)
  \end{align}
\end{subequations}

The big problem with vanilla \textsc{reinforce} lies in variance: if you were to repeatedly collect a small finite number $N$ of samples, estimating each time the gradient based on these samples, you would observe that these estimates of the gradient vary a lot.

That means that the gradient descent step will likely take us away from the goal, and convergence will be very slow, or may not happen.

Additionally, given two sets of sample trajectories differing only by a constant factor, but dofferently centred around a total reward of 0, the basic policy gradient algorithm in \eqref{eq:gradjapprox} may change the parameters very differently. 

%(Note (agravier): I think that this trajectories example in the course is more than a ``variance issue''. This is a fundamental difference in the understanding of the meaning of reward, the information carried by the samples and the assumptions of the algorithm about the unsaid properties of the problem. More research needed here.)

There are two tricks that help with the large varaiance of the samples, b oth of which should always be used because they have no drawbacks.

\subsection{Variance reduction with causality: the ``rewards to go'' trick}

In \eqref{eq:gradjapprox}, the gradient approximation can be improved by making use of causality: the policy at time $t'$ cannot affect the reward at time $t$ when $t < t'$.

\begin{subequations}
  \begin{align}
    \nabla_\theta J(\theta) 
        &\approx \frac{1}{N} \sum_{i=1}^N\left(\left(\sum_{t=1}^T\nabla_\theta\log\pi_\theta\left(\mathbf{a}_t\mid \mathbf{s}_t\right)\right) \sum_{t=1}^T r\left(\mathbf{s}_t,\mathbf{a}_t\right)\right)\\
        &\approx \frac{1}{N} \sum_{i=1}^N\Bigg(\sum_{t=1}^T\Bigg(\underbrace{\nabla_\theta\log\pi_\theta\left(\mathbf{a}_t\mid \mathbf{s}_t\right)}_{\small\llap{\textit{gradient}}\rlap{\textit{ at }t}} \underbrace{\sum_{t'=1}^T r\left(\mathbf{s}_{t'},\mathbf{a}_{t'}\right)}_{\small\llap{\textit{this total }}\rlap{\textit{reward should be computed from time }t}}\Bigg)\Bigg) &\text{by distributivity}\\
        &\approx \frac{1}{N} \sum_{i=1}^N\Bigg(\sum_{t=1}^T\Bigg(\nabla_\theta\log\pi_\theta\left(\mathbf{a}_t\mid \mathbf{s}_t\right) \underbrace{\sum_{t'=t}^T r\left(\mathbf{s}_{t'},\mathbf{a}_{t'}\right)}_{\textit{\small reward to go}}\Bigg)\Bigg) &\text{by causality}
  \end{align}
\end{subequations}

Intuitively, the ``\textbf{rewards to go}'' trick comes from the observation that at time $t$, all past rewards cannot be affected by policy decisions.

We define the ``reward to go'' function $\hat{Q}: \left[\!\left[1,N\right]\!\right] \times \left[\!\left[1,T\right]\!\right] \mapsto \mathbb{R}$ as:

\begin{equation}
  \hat{Q}_{i,t} \doteq \sum_{t'=t}^T r\left(\mathbf{s}_{t'},\mathbf{a}_{t'}\right)
\end{equation}

Therefore, the gradient of the objective function approximation with rewards to go is:

\begin{equation}
  \nabla_\theta J(\theta) \approx \frac{1}{N} \sum_{i=1}^N\sum_{t=1}^T\nabla_\theta\log\pi_\theta\left(\mathbf{a}_t\mid \mathbf{s}_t\right) \hat{Q}_{i,t}
\end{equation}



% References
\small
\bibliographystyle{plain}
\bibliography{bibliography}
\end{document}
